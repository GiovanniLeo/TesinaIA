Consideriamo a questo punto un esempio:


\begin{table}[H]
	\centering
	\begin{tabular}{l l l l l}
		TID & IDCheck & PresDate & ExeDate & RealDate \\
		\hline
		1 & 2 & 2/7/17 & 15/7/17 & 18/7/17\\
		2 & 3 & 3/7/17 & 16/7/17 & 20/7/17\\
		3 & 1 & 2/7/17 & 16/7/17 & 19/7/17\\
		4 & 1 & 5/7/17 & 19/7/17 & 19/7/17\\
		5 & 2 & 8/7/17 & 19/7/17 & 23/7/17\\
	
	\end{tabular}
	\caption{Dataset di esempio.}
	\label{tab:Dataset_di_esempio}
\end{table}

\begin{table}[H]
	\centering
	\begin{tabu}{l l | l l l}
		PairID & ExeDate & IDCheck & PresDate & RealDate \\
		\hline
		2,3 & 0 & 0 & 1 & 1\\
		 \rowfont{\color{gray}}
		4,5 & 0 & 1 & 3 & 4 \\
		\hline
		1,2 & 1 & 0 & 1 & 2\\
		1,3 & 1 & 1 & 0 & 1\\
		\hline
		2,4 & 3 & 0 & 2 & 1\\
		\rowfont{\color{gray}}
		2,5 & 3 & 1 & 5 & 3 \\
		2,3 & 3 & 1 & 3 & 0\\
		\rowfont{\color{gray}}
		4,5 & 3 & 0 & 6 & 4 \\
		\hline
		1,4 & 4 & 0 & 3 & 1\\
		\rowfont{\color{gray}}
		1,5 & 4 & 1 & 6 & 3 \\
		
		
	\end{tabu}
	\caption{Distance Matrix di esempio.}
	\label{tab:Distance_Matrix_di_esempio}
\end{table}
Sia \textbf{$X$}=PresDate, \textbf{$A$}=ExeDate e \textbf{$\epsilon-step$}=1. Allora
\begin{equation*}
      PresDate_{(\leq m-1)} \rightarrow ExeDate_{(\leq next(k))}
\end{equation*}

\begin{table}[H]
	\centering
	\begin{tabu}{l l | l!{\color{red}\vrule} l!{\color{red}\vrule} l}
		PairID & ExeDate & IDCheck & PresDate & RealDate \\
		\hline
		2,3 & 0 & 0 & 1 & 1\\
		\rowfont{\color{gray}}
		4,5 & 0 & 1 & 3 & 4 \\
		\hline
		1,2 & 1 & 0 & 1 & 2\\
		1,3 & 1 & 1 & 0 & 1\\
		\hline
		2,4 & 3 & 0 & 2 & 1\\
		\rowfont{\color{gray}}
		2,5 & 3 & 1 & 5 & 3 \\
		2,3 & 3 & 1 & 3 & 0\\
		\rowfont{\color{gray}}
		4,5 & 3 & 0 & 6 & 4 \\
		\hline
		1,4 & 4 & 0 & \cellcolor{red!25}{3} & 1\\
		\rowfont{\color{gray}}
		1,5 & 4 & 1 & 6 & 3 \\	
	\end{tabu}
	\caption{Prima fase caso base}
	\label{tab:Caso1_1}
\end{table}
Sapendo che il numero 3 evidenziato in tabella \ref{tab:Caso1_1} sia un minimo ottenuto dalla fase di \textit{Minimality} allora per le regole precedenti
\begin{equation*}
PresDate_{(\leq 2)} \rightarrow ExeDate_{(\leq 3)}
\end{equation*}
è una RFD valida. 
\begin{table}[H]
	\centering
	\begin{tabu}{l l | l!{\color{red}\vrule} l!{\color{red}\vrule} l}
		PairID & ExeDate & IDCheck & PresDate & RealDate \\
		\hline
		2,3 & 0 & 0 & 1 & 1\\
		\rowfont{\color{gray}}
		4,5 & 0 & 1 & 3 & 4 \\
		\hline
		1,2 & 1 & 0 & 1 & 2\\
		1,3 & 1 & 1 & 0 & 1\\
		\hline
		2,4 & 3 & 0 & \cellcolor{red!25}{2} & 1\\
		\rowfont{\color{gray}}
		2,5 & 3 & 1 & 5 & 3 \\
		2,3 & 3 & 1 & 3 & 0\\
		\rowfont{\color{gray}}
		4,5 & 3 & 0 & 6 & 4 \\
		\hline
		1,4 & 4 & 0 & 3 & 1\\
		\rowfont{\color{gray}}
		1,5 & 4 & 1 & 6 & 3 \\	
	\end{tabu}
\caption{Seconda fase caso base}
\label{tab:Caso1_2}
\end{table}
Spostandoci a questo punto sul \textit{ClusterID} successivo il minimo è il numero 2 evidenziato in tabella \ref{tab:Caso1_2} (ottenuto come nel caso precedente dalla fase di \textit{Minimality}).
Allora a questo punto si ha che anche 
\begin{equation*}
PresDate_{(\leq 1)} \rightarrow ExeDate_{(\leq 1)}
\end{equation*}
è una RFD valida. Dato ciò non è più possibile generare ulteriori RFD  poichè il cluster successivo ammette un minimo uguale a 0.

%!{\color{red}\vrule}